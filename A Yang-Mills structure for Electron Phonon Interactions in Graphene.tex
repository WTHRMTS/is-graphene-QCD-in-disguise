\documentclass[prb,showpacs,superscriptaddress,titlepage,amsmath,amssymb,twocolumn]{revtex4-1}

\usepackage{graphicx}
\usepackage{epsfig}
\usepackage{subfigure}
\usepackage{hyperref}
\usepackage{epstopdf}
\usepackage{dcolumn}% Align table columns on decimal point
\usepackage{bm}% bold math
\usepackage{color}

\begin{document}

\title{
An SU(3) Yang-Mills Structure for Electron-Phonon Interactions in Graphene
}
\author{J.~M.~Booth}
\email{jamie.booth@rmit.edu.au}
\affiliation{ARC Centre of Excellence in Exciton Science, RMIT University, Melbourne, Australia}

\author{S.~P.~Russo}
\affiliation{ARC Centre of Excellence in Exciton Science, RMIT University, Melbourne, Australia}

\date{\today}

\begin{abstract}
The Gell-Mann matrices are shown to map to spin and charge ordering phonon polarization vectors in three atom sub units of the two-dimensional graphene hexagonal sheets.
\end{abstract}

\maketitle
In a previous study\cite{Booth2020} it was found that an SU(2) Yang-Mills description of electron-phonon interactions in linear systems such as vanadium dioxide can be developed by assuming that the transverse phonons couple to the electron spin via a Rasha-type mechanism. The SU(2) interaction vertex described there has many advantages over the standard U(1) approach to electron-phonon coupling. It contains both charge and spin-ordering and manifests at neighbouring atomic sites and therefore can describe phase transitions in which spin-odering is also present.
In this work we repeat the same process, but we examine the SU(3) gauge group and find that it also describes charge and spin ordering in a low dimensional structure, however unlike linear systems as per SU(2) it describes hexagonal systems, such as graphene.

\bibliography{library.bib}

\end{document}